%%%%%%%%%%%%%%%%%%%%%%%%%%%%%%%%%%%%%%%%%%%%%%%%%%%%%%%%%%%%%%%%%%%%%%%%%%%%%%%%%%
%
% Author: Alex M. <alex@codefu.ca>
% https://github.com/Alexstre/simple-latex-template/
%
% This program is free software: you can redistribute it and/or modify
% it under the terms of the GNU General Public License as published by
% the Free Software Foundation, either version 3 of the License, or
% (at your option) any later version.
% 
% This program is distributed in the hope that it will be useful,
% but WITHOUT ANY WARRANTY; without even the implied warranty of
% MERCHANTABILITY or FITNESS FOR A PARTICULAR PURPOSE.  See the
% GNU General Public License for more details.
%
% You should have received a copy of the GNU General Public License
% along with this program.  If not, see <http://www.gnu.org/licenses/>.
%
% Mock data found in the examples folder generated with https://www.mockaroo.com
%
%%%%%%%%%%%%%%%%%%%%%%%%%%%%%%%%%%%%%%%%%%%%%%%%%%%%%%%%%%%%%%%%%%%%%%%%%%%%%%%%%%

%----------------------------------------------------------------------------------------
%	CONFIGURATIONS
%----------------------------------------------------------------------------------------

\documentclass{article}
\usepackage{amsmath}
\usepackage{hyperref} % To include links with \url

% Encoding / corrections
\usepackage[utf8]{inputenc}
\usepackage{lmodern} % Use the modern fonts
\usepackage[T1]{fontenc}

% Use for placeholder content
\usepackage{blindtext} % Can be commented or removed

% Uncomment for French
%\usepackage[french]{babel}
\usepackage[english]{babel}

% Graphs and plots
\usepackage{tikz}
\usepackage{pgfplots}
\usepackage{graphicx}
\usepackage{float}

% Creates tables from CSV files
\usepackage{booktabs}
\usepackage{pgfplotstable}

% Removes indent on paragraphs
\setlength\parindent{0pt}

% Settings for plots
\pgfplotsset{compat=newest}
\usepgfplotslibrary{units}

% Geometry settings (reduces margins)
\usepackage[a4paper,bindingoffset=0.2in,left=1in,right=1in,top=1in,bottom=1in,footskip=.25in]{geometry}

% Sets the path to the CSV files, use \datadir anywhere
\newcommand*{\datadir}{example}

%----------------------------------------------------------------------------------------
%	TABLE DEFINITIONS
%----------------------------------------------------------------------------------------
\pgfplotstableread[col sep=comma]{\datadir/cities.csv}\citiestable

%----------------------------------------------------------------------------------------
%	INFORMATION
%----------------------------------------------------------------------------------------

\title{Tile should \\ Come here} % Title

\author{John \textsc{Smith}}

\begin{document}

\clearpage\maketitle
\thispagestyle{empty}

\begin{center}
\begin{tabular}{l r}
Date: & June 10th 2015 \\ 
\end{tabular}
\end{center}

\newpage

%----------------------------------------------------------------------------------------
%	Table of Contents
%----------------------------------------------------------------------------------------
\tableofcontents
\newpage


%----------------------------------------------------------------------------------------
%	A section without much content
%----------------------------------------------------------------------------------------

\section{Section 1}
\blindtext[3]

%----------------------------------------------------------------------------------------

%----------------------------------------------------------------------------------------
%	A section with subsection and a plot (data from csv files)
%----------------------------------------------------------------------------------------

\section{Another section}
\blindtext[2]

\subsection{Lorem ipsum dolor sit amet}
% Plot
\begin{figure}[H]
  \begin{center}
    \begin{tikzpicture}
      \begin{axis}[
          width=\linewidth,
          grid=major,
          grid style={dashed,gray!30},
          % Uncomment for log scale
          %xmode=log, log basis x=10,
          %ymode=log, log basis y=10,
          xlabel=id, % Label for X axis
          ylabel=value, % Label for Y axis
          legend style={at={(0.5,-0.2)},anchor=north}, % Legend Style
          x tick label style={rotate=90,anchor=east}
        ]
        % Plots. Each Multiple \addplot show on the same plot
        \addplot table[x=id,y=value,col sep=comma] {\datadir/plotdata.csv}; 
        \addplot table[x=id,y=value,col sep=comma] {\datadir/plotdata2.csv}; 
        \addplot table[x=id,y=value,col sep=comma] {\datadir/plotdata3.csv};
        \legend{Legend goes here, Second, Third} % Comma-separated for each plot
      \end{axis}
    \end{tikzpicture}
    \caption{Caption goes here}
  \end{center}
\end{figure}

\newpage

% Another subsection
\subsection{Duis dapibus lacinia lacinia}

%----------------------------------------------------------------------------------------
%	A table: See the definition in the TABLE DEFINITIONS section above
%----------------------------------------------------------------------------------------
\begin{center}
\pgfplotstabletypeset[
string type,
every head row/.style={before row={\toprule},after row=\midrule},
every last row/.style={after row={\toprule}},
]\citiestable
\end{center}
%----------------------------------------------------------------------------------------

%----------------------------------------------------------------------------------------
%	Another section example with sub and subsub sections.
%----------------------------------------------------------------------------------------

\section{Ante ipsum primis}
\blindtext[1]
\subsection{Ipsum dolor sit}
\blindtext[1]
\subsubsection{Lorem ipsum dolor sit amet}
\blindtext[1]
\blindenumerate[5]

%----------------------------------------------------------------------------------------

\end{document}